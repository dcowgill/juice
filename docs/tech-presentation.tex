\documentclass{beamer}


\mode<presentation>
{
  \usetheme{Pittsburgh}
}


\usepackage[english]{babel}
\usepackage[latin1]{inputenc}
\usepackage{listings}

\title{A Whirlwind Tour Of juice}


\author{Graham Lowe}

\date{\today}



\beamerdefaultoverlayspecification{<+->}


\begin{document}

\begin{frame}
  \titlepage
\end{frame}

\begin{frame}{What's In A Name?}

  \begin{itemize}
  \item
    \textbf{j}avascript
  \item
    \textbf{u}ser
  \item
    \textbf{i}nterface
  \item
    \textbf{c}ompiled
  \item
    \textbf{e}nvironment
  \end{itemize}
\end{frame}

\begin{frame}{Project Goals}
  \begin{itemize}
  \item
    Decouple front-end and back-end development.
  \item
    Encourage component-based design.
  \item
    Make rich interface development easier.
  \end{itemize}
\end{frame}

\begin{frame}{Architecture}
  \begin{centering}
    \pgfimage[height=7.5cm]{architecture.png}
    \par
  \end{centering}
\end{frame}

\begin{frame}{Page}
  \begin{itemize}
  \item JavaScript object.
  \item URL, layout, widgets, stylesheets, etc.
  \item Defined using \texttt{juice.page.define}.
  \end{itemize}
\end{frame}

\begin{frame}[fragile]{Example Definition}
\begin{verbatim}
juice.page.define(
    {name: 'front',
     title: 'Login',
     path: '/',
     layout: proj.layouts.one_column,
     widget_packages: ['front_page'],
     init_widgets: function(args) {
         return {a: [proj.widgets.front_page.login()]};
     }
    });
\end{verbatim}
\end{frame}

\begin{frame}[fragile]{Smart URL Argument Handling}

Define what URL information your page requires

\begin{verbatim}
juice.page.define(
    {name: 'profile',
     path: '/[[username \\w+]]/',
     parameters: ['username'],
    // rest omitted
\end{verbatim}

and juice will enforce it

\begin{verbatim}
// Good
good = proj.pages.profile.url({username: 'graham'});

// Throws error!
bad = proj.pages.user_profile.url();
\end{verbatim}

\end{frame}


\begin{frame}{Layout}
  \begin{itemize}
  \item Describe the panels on page.
  \item Widgets are placed into panels.
  \item Rely on CSS for positioning of panels.
  \item Defined using \texttt{juice.layout.define}.
  \end{itemize}
\end{frame}

\begin{frame}[fragile]{Example Definition}
\begin{verbatim}
juice.layout.define('one_column', {a: null});
\end{verbatim}
\end{frame}

\begin{frame}{Widgets}
  \begin{itemize}
  \item Encapsulate behavior and presentation.
  \item Related widgets are organized into packages.
  \item Well-defined life-cycle.
  \item Defined using \texttt{juice.page.define}
  \end{itemize}
\end{frame}

\begin{frame}[fragile]{Example Definition}
\begin{verbatim}
juice.widget.define(
    'hello',
    function(that, my, spec) {
        my.render = templates.hello({name: spec.name});
    });
\end{verbatim}
\end{frame}

\begin{frame}{Templates}
  \begin{itemize}
  \item Automatic safety.
  \item Simple syntax: HTML, template tags, and modifiers.
  \item May contain widgets and other templates.
  \item Compiled to efficient JavaScript functions.
  \item Detect syntax errors at compile time.
  \end{itemize}
\end{frame}

\begin{frame}[fragile]{A template definition}
\begin{verbatim}
<ul>
  
  <li>
    <a href="#{{tab.id}}"><span>{{tab.title}}</span></a>
  </li>
  
</ul>

<div id="{{tab.id}}">{{tab.widget}}</div>

\end{verbatim}
\end{frame}

\begin{frame}{RPCS}
  \begin{itemize}
  \item Represent a remote service.
  \item Map local name to remote service name.
  \item Request parameters and response parameters are validated.
  \item Related RPCs are organized into packages.
  \end{itemize}
\end{frame}

\begin{frame}[fragile]{Example}
Here's an example RPC definition:
\begin{verbatim}
juice.rpc.define({name: 'delete_many',
                  service_name: 'notes_delete_many',
                  req_spec: {ids: ['string']},
                  rsp_spec: null});
\end{verbatim}
\end{frame}

\begin{frame}[fragile]{Example invocation}

Here's an example invocation of the RPC (note the callback):
\begin{verbatim}
proj.rpcs.notes.delete_many(
    {ids: ['1', '2', '3', '4']},
    function(response) {
        // This will get executed when the RPC gets a
        // response from the server
        juice.util.message.notice(
            'A bunch of notes were deleted');
    });
\end{verbatim}
\end{frame}

\begin{frame}{RPC Proxies}
  \begin{itemize}
  \item Layer of indirection adds flexibility.
  \item Transports: json, xml, REST, jsonp.
  \item Boxcarring.
  \item Mocking.
  \item Different service providers.
  \end{itemize}
\end{frame}

\begin{frame}{Q \& A}
  Any questions before we start demo?
\end{frame}

\end{document}
